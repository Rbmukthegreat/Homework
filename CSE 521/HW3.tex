\documentclass[12pt]{article}
\usepackage[margin=1in]{geometry}

% Start of preamble
%==========================================================================================%
% Required to support mathematical unicode
\usepackage[warnunknown, fasterrors, mathletters]{ucs}
\usepackage[utf8x]{inputenc}

\usepackage[dvipsnames,table,xcdraw]{xcolor}
\usepackage{hyperref} 
\hypersetup{
	colorlinks=true,
	linkcolor=blue,
	filecolor=magenta,      
	urlcolor=cyan,
	pdfpagemode=FullScreen
}

% Standard mathematical typesetting packages
\usepackage{amsmath,amssymb,amscd,amsthm,amsxtra, pxfonts}
\usepackage{mathtools,mathrsfs,dsfont,xparse}

% Symbol and utility packages
\usepackage{cancel, textcomp}
\usepackage[mathscr]{euscript}
\usepackage[nointegrals]{wasysym}
\usepackage{apacite}

% Extras
\usepackage{physics}  
\usepackage{tikz-cd} 
\usepackage{microtype}
\usepackage{enumitem}
\usepackage{titling}
\usepackage{graphicx}

% Fancy theorems due to @intuitively on discord
\usepackage{mdframed}
\newmdtheoremenv[
backgroundcolor=NavyBlue!30,
linewidth=2pt,
linecolor=NavyBlue,
topline=false,
bottomline=false,
rightline=false,
innertopmargin=10pt,
innerbottommargin=10pt,
innerrightmargin=10pt,
innerleftmargin=10pt,
skipabove=\baselineskip,
skipbelow=\baselineskip
]{mytheorem}{Theorem}

\newenvironment{theorem}{\begin{mytheorem}}{\end{mytheorem}}

\newtheorem{corollary}{Corollary}
\newtheorem{lemma}{Lemma}

\newtheoremstyle{definitionstyle}
{\topsep}%
{\topsep}%
{}%
{}%
{\bfseries}%
{.}%
{.5em}%
{}%
\theoremstyle{definitionstyle}
\newmdtheoremenv[
backgroundcolor=Violet!30,
linewidth=2pt,
linecolor=Violet,
topline=false,
bottomline=false,
rightline=false,
innertopmargin=10pt,
innerbottommargin=10pt,
innerrightmargin=10pt,
innerleftmargin=10pt,
skipabove=\baselineskip,
skipbelow=\baselineskip,
]{mydef}{Definition}
\newenvironment{definition}{\begin{mydef}}{\end{mydef}}

\newtheorem*{remark}{Remark}

\newtheorem*{example}{Example}

% Common shortcuts
\def\mbb#1{\mathbb{#1}}
\def\mfk#1{\mathfrak{#1}}

\def\bN{\mbb{N}}
\def \C{\mbb{C}}
\def \R{\mbb{R}}
\def\bQ{\mbb{Q}}
\def\bZ{\mbb{Z}}
\def \cph{\varphi}
\renewcommand{\th}{\theta}
\def \ve{\varepsilon}
\newcommand{\mg}[1]{\| #1 \|}

% Often helpful macros
\newcommand{\floor}[1]{\left\lfloor#1\right\rfloor}
\newcommand{\ceil}[1]{\left\lceil#1\right\rceil}
\renewcommand{\qed}{\hfill\qedsymbol}
\renewcommand{\P}{\mathrm{Pr}\qty}
\newcommand{\E}{\mathbb{E}\qty}
\newcommand{\Var}{\mathrm{Var}\qty}

% Sets
\usepackage{braket}

% End of preamble
%==========================================================================================%

% Start of commands specific to this file
%==========================================================================================%

%==========================================================================================%
% End of commands specific to this file

\title{CSE 521 HW3}
\date{\today}
\author{Rohan Mukherjee}

\begin{document}
	\maketitle
	\begin{enumerate}[leftmargin=\labelsep]
		\item I claim that $h_A = h_B$ iff the minimum value of $h$ among $A \cup B$ occurs in $A \cap B$. The reverse direction is clear, so suppose that the minimum value of $h$ was in $(A \cup B) \setminus (A \cap B)$. Then $h_A < h_B$ or vice versa, a contradiction. Thus, 
		\begin{align*}
			\P[h_A = h_B] = \frac{|A \cap B|}{|A \cup B|} = J(A, B)
		\end{align*}
	
		\item We see that
		\begin{align*}
			\E[Y] = \int_0^1 \P[Y \geq x]dx = \int_0^1 \prod_{i=1}^n \P[X_i \geq x]dx = \int_0^1 (1-x)^ndx = \int_0^1 x^ndx = \frac{1}{n+1}
		\end{align*}
		Also, since $Y^2 = \min \Set{X_1, \ldots, X_n}^2 = \min\Set{X_1^2, \ldots, X_n^2}$, we have that 
		\begin{align*}
			\E[Y^2] = \int_0^1 \P[Y^2 \geq x]dx = \int_0^1 \prod_{i=1}^n \P[X_i \geq \sqrt{x}]dx = \int_0^1 (1-\sqrt{x})^n dx = \frac{2}{(n+1)(n+2)} \leq \frac{2}{(n+1)^2}
		\end{align*}
		Thus $\Var(Y) = \E[Y^2] - \E[Y]^2 \leq \frac{1}{(n+1)^2}$.
		
		\item \begin{enumerate}
			\item At the end of the stream $Y$ is the minimum of $F_0$ independent uniformly distributed r.v.s (since $h(i)$ is a uniform r.v. in $[0,1]$ for each $i$) in $[0, 1]$, and hence by question 2 its expectation is $\frac{1}{F_0+1}$. So $\frac{1}{\E[Y]}-1 = F_0$.
			
			\item We also see that $\Var(Y) \leq \frac{1}{(F_0+1)^2}$. Thus $t(Y) \leq 1$. So $Y$ is an unbiased estimator of $\frac{1}{F_0+1}$ with relative variance at most 1, and hence we can approximate $\frac{1}{F_0+1}$ within a $1 \pm \ve$ multiplicative factor using only $k = O\qty(\frac{1}{\ve^2}\log \frac1{0.1})=O\qty(\frac{1}{\ve^2})$ independent samples of $Y$ with probability $1-0.1 = 9/10$. I claim that finding $\frac{1}{F_0+1}$ within a $1 \pm \ve$ multiplicative error is enough to find $F_0$ within a $1\pm 4\ve$ multiplicative error. This is true because
			\begin{align*}
				\frac{1-\ve}{F_0+1} \leq Y \leq \frac{1+\ve}{F_0+1} \iff \frac{F_0-\ve}{1+\ve} \leq \frac{1}{Y} - 1 \leq \frac{F_0+\ve}{1-\ve}
			\end{align*}
			And because 
			\begin{align*}
				\frac{F_0+\ve}{1-\ve} = F_0 \frac{1+\ve/F_0}{1-\ve} \leq F_0\frac{1+\ve}{1-\ve} \leq F_0(1+\ve)(1+2\ve) = F_0(1+3\ve+2\ve^2) \leq F_0(1+4\ve)
			\end{align*}
			Where the first inequality holds since $F_0 \geq 1$, the second since $\sum_{k=1}^\infty \ve^k \leq \ve/(1-\ve) \leq 2\ve$ for $\ve < 1/2$, and the last when $\ve$ is sufficiently small.
		\end{enumerate}
	\end{enumerate}
\end{document}
