\documentclass[12pt]{article}
\usepackage[margin=1in]{geometry}

% Start of preamble
%==========================================================================================%
% Required to support mathematical unicode
\usepackage[warnunknown, fasterrors, mathletters]{ucs}
\usepackage[utf8x]{inputenc}

\usepackage[dvipsnames,table,xcdraw]{xcolor}
\usepackage{hyperref} 
\hypersetup{
	colorlinks=true,
	linkcolor=blue,
	filecolor=magenta,      
	urlcolor=cyan,
	pdfpagemode=FullScreen
}

% Standard mathematical typesetting packages
\usepackage{amsmath,amssymb,amscd,amsthm,amsxtra, pxfonts}
\usepackage{mathtools,mathrsfs,dsfont,xparse}

% Symbol and utility packages
\usepackage{cancel, textcomp}
\usepackage[mathscr]{euscript}
\usepackage[nointegrals]{wasysym}
\usepackage{apacite}

% Extras
\usepackage{physics}  
\usepackage{tikz-cd} 
\usepackage{microtype}
\usepackage{enumitem}
\usepackage{titling}
\usepackage{graphicx}
\usepackage{bbm}

% Fancy theorems due to @intuitively on discord
\usepackage{mdframed}
\newmdtheoremenv[
backgroundcolor=NavyBlue!30,
linewidth=2pt,
linecolor=NavyBlue,
topline=false,
bottomline=false,
rightline=false,
innertopmargin=10pt,
innerbottommargin=10pt,
innerrightmargin=10pt,
innerleftmargin=10pt,
skipabove=\baselineskip,
skipbelow=\baselineskip
]{mytheorem}{Theorem}

\newenvironment{theorem}{\begin{mytheorem}}{\end{mytheorem}}

\newtheorem{corollary}{Corollary}
\newtheorem{lemma}{Lemma}

\newtheoremstyle{definitionstyle}
{\topsep}%
{\topsep}%
{}%
{}%
{\bfseries}%
{.}%
{.5em}%
{}%
\theoremstyle{definitionstyle}
\newmdtheoremenv[
backgroundcolor=Violet!30,
linewidth=2pt,
linecolor=Violet,
topline=false,
bottomline=false,
rightline=false,
innertopmargin=10pt,
innerbottommargin=10pt,
innerrightmargin=10pt,
innerleftmargin=10pt,
skipabove=\baselineskip,
skipbelow=\baselineskip,
]{mydef}{Definition}
\newenvironment{definition}{\begin{mydef}}{\end{mydef}}

\newtheorem*{remark}{Remark}

\newtheorem*{example}{Example}

% Common shortcuts
\def\mbb#1{\mathbb{#1}}
\def\mfk#1{\mathfrak{#1}}

\def\bN{\mbb{N}}
\def \C{\mbb{C}}
\def \R{\mbb{R}}
\def\bQ{\mbb{Q}}
\def\bZ{\mbb{Z}}
\def \cph{\varphi}
\renewcommand{\th}{\theta}
\def \ve{\varepsilon}
\newcommand{\mg}[1]{\| #1 \|}

% Often helpful macros
\newcommand{\floor}[1]{\left\lfloor#1\right\rfloor}
\newcommand{\ceil}[1]{\left\lceil#1\right\rceil}
\renewcommand{\qed}{\hfill\qedsymbol}
\renewcommand{\P}{\mathbb{P}\qty}
\newcommand{\E}{\mathbb{E}\qty}
\renewcommand{\O}{O\qty}

% Sets
\usepackage{braket}

% Code
\usepackage{listings}
\usepackage{color}

\definecolor{dkgreen}{rgb}{0,0.6,0}
\definecolor{gray}{rgb}{0.5,0.5,0.5}
\definecolor{mauve}{rgb}{0.58,0,0.82}

\lstset{frame=tb,
	language=Python,
	aboveskip=3mm,
	belowskip=3mm,
	showstringspaces=false,
	columns=flexible,
	basicstyle={\small\ttfamily},
	numbers=none,
	numberstyle=\tiny\color{gray},
	keywordstyle=\color{blue},
	commentstyle=\color{dkgreen},
	stringstyle=\color{mauve},
	breaklines=true,
	breakatwhitespace=true,
	tabsize=3
}

% End of preamble
%==========================================================================================%

% Start of commands specific to this file
%==========================================================================================%

\newcommand{\lp}[1]{\mg{#1}_p}

%==========================================================================================%
% End of commands specific to this file

\title{CSE 521 HW4}
\date{\today}
\author{Rohan Mukherjee}

\begin{document}
	\maketitle
	\begin{enumerate}[leftmargin=\labelsep]
		\item \begin{enumerate}
			\item Notice for $\alpha > 0$ and some fixed $y \in \R^n$, 
			\begin{align*}
				\P[\sum_{i=1}^n y_i \alpha Z_i \geq x] = \P[\sum_{i=1}^n y_i Z_i \geq \frac{x}{\alpha}] = \P[\mg{y}_p Z \geq \frac{x}{\alpha}] = \P[\mg{y}_p \alpha Z \geq x]
			\end{align*}
		
			\item Since the pdf is continuous and bounded, we see that $\int_{1-\ve}^1 p(x)dx = p(c_\ve) \cdot \ve$ for some suitable choice of $c_\ve \in (1-\ve, 1)$. The pdf is also symmetric, since $-Z$ has the same distribution as $\mg{-1}_p Z = Z$. Thus 
			\begin{align*}
				\P[-1 + \ve < Z < 1 - \ve] = 2\qty(\int_0^1 p(x)dx - \int_{1-\ve}^{1} p(x)dx) = \frac 12 - 2p(c_\ve) \ve \leq \frac12 - c \ve
			\end{align*}
		
			Similarly $\int_{1}^{1+\ve} p(x)dx = p(d_\ve) \cdot \ve$ for a suitable choice of $c_\ve$. Thus,
			\begin{align*}
				\P[-1-\ve < Z < 1 + \ve] = 2\qty(\int_0^1 p(x)dx + \int_{1}^{1+\ve} p(x)dx) = \frac12 + 2p(d_\ve)\ve \geq \frac12 + c \ve
			\end{align*}
			Since $p(1) > 0$ we can find a sufficiently small neighborhood so that $p(x) > p(1)/2=c$ for all $x$ close to 1. 
		
			\item Notice that $y_i = (Px)_i = \sum_j x_j Z_{ij}$ just by matrix multiplication. Define $a_i = |y_i| < \lp{x}(1-\ve)$ and $b_i = |y_i| > \lp{x}(1+\ve)$. Clearly $\mbb E a_i = \P[|y_i| < \lp{x}(1-\ve)] = \P[|Z| < 1-\ve] \leq \frac12 - c\ve$, and $\mbb E b_i \geq \frac12 + c\ve$. Let $S_\ell = \sum_{i=1}^\ell a_i$. Notice that
			\begin{align*}
				\P[S_\ell \geq \ell/2] = \P[S_\ell - \mbb E S_\ell \geq \ell/2 - \mbb E S_\ell] \leq \P[|S_\ell - \mbb E S_\ell| \geq 2C\ve l] \leq 2e^{-2C^2 \ve^2 l}
			\end{align*}
			Similarly if we let $T_\ell = \sum_{i=1}^\ell b_i$ we get
			\begin{align*}
				\P[T_\ell \geq \ell/2] \leq 2e^{-2C^2\ve^2 \ell}
			\end{align*}
			Now we want 
			\begin{align*}
				\P[S_\ell \leq \ell/2, \; T_\ell \leq \ell/2] \geq 1 - 4e^{-2C^2\ve^2\ell} \geq 1 - \delta
			\end{align*}
			It suffices to choose $\ell = \frac{\log(4/\delta)}{C^2\ve^2} = \O(\frac{\log 1/\delta}{\ve^2})$. Notice that if both $S_l \leq \ell/2$ and $T_l \leq \ell/2$, and $\ell$ is odd, then we certainly have that the median is in $[\lp{x}(1-\ve), \lp{x}(1+\ve)]$.
			
			\item Notice that, if $Z \sim \mathrm{Cauchy}(0, 1)$,
			\begin{align*}
				\P[1+\ve < Z < 1-\ve] = \frac1\pi \arctan(1-\ve) - \frac1\pi \arctan(1+\ve) \approx \frac{1}{2\pi} 2\ve = \frac{\ve}{\pi}
			\end{align*}
			By the mean value theorem. So we must use $\ell = \pi^2\log(4/0.01)/\ve^2$ to get a $1\pm \ve$ approximation with probability $99/100$. I shall choose $\ve = 0.01$ as well, so we need $\ell \approx 591334$. Here is my code:
			\begin{lstlisting}
				import numpy as np
				from numpy import random as rand
				
				x = [int(x) for x in open("p4.in", "r")]
				P = rand.standard_cauchy((591334, len(x)))
				y = np.absolute(P.dot(x))
				print(np.median(y))
			\end{lstlisting}
			I get a value of 4788.158226259304.
		\end{enumerate}
	
		\item \begin{enumerate}
			\item For $a = 0$, 
		\begin{align*}
			\P[\floor{\frac{-s}{w}} = \floor{\frac{b-s}{w}}] = \P[-1 = \floor{\frac{b-s}{w}}]
		\end{align*}
		This happens precisely when $-w < b-s < 0$, or $b < s < b+w$. $s$ can only be in this range if $|b| < w$, else the probability is 0. If $b$ is positive then we need $b < s < w$, which happens with probability $(w-b)/w = 1 - b/w$. Similarly if it is negative we need $1+b/w$. So the probability is $\max \Big\{0, 1 - \frac{|b|}{w}\Big\}$. Similarly, if 
		\begin{align*}
			\floor{\frac{a-s}{w}} = \floor{\frac{b-s}{w}}
		\end{align*}
		Then,
		\begin{align*}
			\frac{a-s}{w} - 1 &\leq \frac{b-s}{w} \leq \frac{a-s}{w} + 1 \\
			\iff a - s - w &\leq b - s \leq a - s + w \\
			\iff a - w &\leq b \leq a + w \\
			\iff |b-a| &\leq w
		\end{align*}
		So if $|b-a| > w$ the probability is already 0. Now WLOG $b > a$. Since $\floor{x} = \floor{y}$ iff $\floor{x-k} = \floor{y-k}$ (for integer $k$), and since $|b-a| < 1$, we can assume that $0 < a,b < 2$. Now either $(a, b) \subset (0, 1)$ or $1 \in (a, b)$. In the first case, the only way the floors won't be equal is if $s \in (a, b)$, since then $\floor{a-s} = -1$ while $\floor{b-s} = 0$ (Else they are either both 0 or both -1). In the second case, the only way they won't be equal is if $s$ takes away too little from $b$, or too much from $a$. That is, if $s \in (0, b-1)$, or $s \in (a, 1)$. The probability this bad event happens is just $b-a$. Similarly, in the general case, for $k$ an integer, $\floor{x/w} = \floor{y/w}$ iff $\floor{x/w + kw/w} = \floor{y/w + kw/w}$. We can again assume that $0 < a,b < 2w$. The exact same logic as before works again but just having dividing by $w$ everywhere (indeed: we are just stretching the number line by a factor of $1/w$), so we can conclude the probability is $\max \Big\{0, 1 - \frac{|a-b|}{w}\Big\}$.
		
		\item We see that
		\begin{align*}
			\P[h(p)=h(q)] &= \P[\bigcap_{i=1}^d \floor{\frac{p_i-s_i}{w}} = \floor{\frac{q_i-s_i}{w}}] = \prod_{i=1}^d \qty(1-\frac{\alpha_i}{w}) \\
			&\approx \exp(-\frac1w\sum_{i=1}^d \alpha_i) = \exp(-\frac1w \mg{p-q}_1)
		\end{align*}
		So if $d(p, q) = \mg{p-q}_1 \leq r$, we have
		\begin{align*}
			\P[h(p)=h(q)] \geq \exp(-\frac rw)
		\end{align*}
		And when $d(p, q) \geq c \cdot r$, we have
		\begin{align*}
			\P[h(p)=h(q)] \leq \exp(-\frac{cr}{w})
		\end{align*}
		So this hash function is $(r, c \cdot r, e^{-r/w}, e^{-cr/w})$-sensitive.
		\end{enumerate}
	\end{enumerate}
\end{document}
