\documentclass[12pt]{article}
\usepackage[margin=1in]{geometry}

% Start of preamble
%==========================================================================================%
% Required to support mathematical unicode
\usepackage[warnunknown, fasterrors, mathletters]{ucs}
\usepackage[utf8x]{inputenc}

\usepackage[dvipsnames,table,xcdraw]{xcolor} % colors
\usepackage{hyperref} % links
\hypersetup{
	colorlinks=true,
	linkcolor=blue,
	filecolor=magenta,      
	urlcolor=cyan,
	pdfpagemode=FullScreen
}

% Standard mathematical typesetting packages
\usepackage{amsmath,amssymb,amscd,amsthm,amsxtra, pxfonts}
\usepackage{mathtools,mathrsfs,dsfont,xparse}

% Symbol and utility packages
\usepackage{cancel, textcomp}
\usepackage[mathscr]{euscript}
\usepackage[nointegrals]{wasysym}
\usepackage{apacite}

% Extras
\usepackage{physics}  % Lots of useful shortcuts and macros
\usepackage{tikz-cd}  % For drawing commutative diagrams easily
\usepackage{microtype}  % Minature font tweaks
%\usepackage{pgfplots} % plots

\usepackage{enumitem}
\usepackage{titling}

\usepackage{graphicx}

% Fancy theorems due to @intuitively on discord
\usepackage{mdframed}
\newmdtheoremenv[
backgroundcolor=NavyBlue!30,
linewidth=2pt,
linecolor=NavyBlue,
topline=false,
bottomline=false,
rightline=false,
innertopmargin=10pt,
innerbottommargin=10pt,
innerrightmargin=10pt,
innerleftmargin=10pt,
skipabove=\baselineskip,
skipbelow=\baselineskip
]{mytheorem}{Theorem}

\newenvironment{theorem}{\begin{mytheorem}}{\end{mytheorem}}

\newtheorem{corollary}{Corollary}
\newtheorem{lemma}{Lemma}

\newtheoremstyle{definitionstyle}
{\topsep}%
{\topsep}%
{}%
{}%
{\bfseries}%
{.}%
{.5em}%
{}%
\theoremstyle{definitionstyle}
\newmdtheoremenv[
backgroundcolor=Violet!30,
linewidth=2pt,
linecolor=Violet,
topline=false,
bottomline=false,
rightline=false,
innertopmargin=10pt,
innerbottommargin=10pt,
innerrightmargin=10pt,
innerleftmargin=10pt,
skipabove=\baselineskip,
skipbelow=\baselineskip,
]{mydef}{Definition}
\newenvironment{definition}{\begin{mydef}}{\end{mydef}}

\newtheorem*{remark}{Remark}

\newtheorem*{example}{Example}

% Common shortcuts
\def\mbb#1{\mathbb{#1}}
\def\mfk#1{\mathfrak{#1}}

\def\bN{\mbb{N}}
\def \C{\mbb{C}}
\def \R{\mbb{R}}
\def\bQ{\mbb{Q}}
\def\bZ{\mbb{Z}}
\def \cph{\varphi}
\renewcommand{\th}{\theta}
\def \ve{\varepsilon}
\newcommand{\mg}[1]{\| #1 \|}

% Often helpful macros
\newcommand{\floor}[1]{\left\lfloor#1\right\rfloor}
\newcommand{\ceil}[1]{\left\lceil#1\right\rceil}
\renewcommand{\qed}{\hfill\qedsymbol}
\renewcommand{\ip}[2]{\langle #1, #2 \rangle}
\newcommand{\seq}[2]{\qty(#1_#2)_{#2=1}^{\infty}}

% Sets
\usepackage{braket}

% End of preamble
%==========================================================================================%

% Start of commands specific to this file
%==========================================================================================%

%==========================================================================================%
% End of commands specific to this file

\title{Template}
\date{\today}
\author{Rohan Mukherjee}

\begin{document}
	\maketitle
	\begin{enumerate}[leftmargin=\labelsep]
		\item We factorize $20,000,000 = 2 \cdot 10^7 = 2^8 \cdot 5^7$. A perfect square has all factors to an even power, which leaves the possibilities of 0, 2, 4, 6, and 8 for the power of 2, and 0, 2, 4, and 6 for the power of 5. This yields $5 \cdot 4 = 20$ factors.
		
		\item We proceed by assigning each day a nonempty subset of $\Set{\mathrm{Alice}, \mathrm{Bob}, \mathrm{Charlie}}$. There are of course $2^3 - 2^0 = 7$ subsets of this form, and we are assigning each day one of these, yielding $7^3$ ways.
		
		\item If rotations and reflections were different there would be $7!$ ways of doing this. However, each bracelet has 7 rotations and 1 reflection, so we are overcounting by a factor of 8. This yields $7!/8$ to be our final answer.
		
		\item The idea is that we need two ones surrounding any 0. So we place all the ones and then we place the zeros in between them. This gives 9 spots to place the zeros (the 7 in between and the 2 on the outside), which we need to choose 4 to be the positions of our zeros, yielding our final answer of ${9 \choose 4}$.
		
		\item If we pretend the T's and E's are separate, this would give $4$ choices for the first letter (S, T1, T2, L) and then we have to permute the remaining letters, giving us $4 \cdot 6!$. But we are overcounting by a factor of $2! \cdot 2!$, since we have 2 E's and 2 T's, so our final answer is $4 \cdot 6! / (2!)^2$.
		
		\item The trick with these kinds of problems where you have a condition like ``at least 1'' ``at most 1'' is just to divide into cases. The chores could be split up as 4, 1, 1; 3, 2, 1; or 2, 2, 2. For the first case there are 3 ways to choose the person who gets 4 chores, and then we just have to choose 4 chores from 6, and 2 from 1, giving $3 \cdot {6 \choose 4} \cdot {2 \choose 1}$ ways, the second case there is $3!$ ways to pick who gets the 3, 2, 1 respectively, then get ${6 \choose 3} \cdot {3 \choose 2}$ ways to give the chores out, and for the last way we just have ${6 \choose 2} \cdot {4 \choose 2}$ ways to give the chores out. This gives us a final answer of
		\begin{align*}
			3 \cdot {6 \choose 4, 2, 1} + 3! \cdot {6 \choose 3, 2, 1} + {6 \choose 2, 2, 2}
		\end{align*}
	
		\item We have 16 indistinguishable dollars which we want to give to four indistinguishable cousins, where each cousin gets at least 2 dollars. So give each cousin 2 dollars first, so there are 8 dollars remaining, which we use stars and bars on to get
		\begin{align*}
			{8 + 4 - 1 \choose 4 - 1} = {11 \choose 3}
		\end{align*}
	
		\item We just have to divide into cases. We have $a + b + c + d  = 0, \ldots, a + b + c + d = 6$. Now we can just use stars and bars--for the first case, we have 0 indistinguishable balls that we need to put in 4 distinguishable boxes, yielding ${0 + 4 - 1 \choose 4 - 1}$ ways. Similarly, for $a + b + c + d = i$ for $1 \leq i \leq 6$, we get ${i + 4 - 1 \choose 4 - 1}$ ways. This gives us a total answer of
		\begin{align*}
			\sum_{i=0}^6 {i + 3 \choose 3}
		\end{align*}
	
		\item Either we have a pomegranate or we don't. In the first case, we need to distribute 4 identical fruits into 4 distinguishable fruit boxes, giving us ${4 + 4 - 1 \choose 3}$ ways. In the second, we need to distribute 5 identical fruits into 4 distinguishable boxes, yielding ${5 + 4 - 1 \choose 3}$ ways. This gives us a total answer of
		\begin{align*}
			{7 \choose 3} + {8 \choose 3}
		\end{align*}
	
		\item The coefficient of $xy^2$ in $(3-x+2y)^5$ by the multinomial theorem we just need to pick 1 $x$, $2$ $y$'s, and the remaining 3 3's, which gives us ${6 \choose 1, 2, 3} \cdot 3^3 \cdot (-1)^1 \cdot 2^2$.
		
		\item The power series expansion of $\sqrt{1+2x^2}$ is just $\sum_{i=0}^\infty {\frac12 \choose i} 2^i x^{2i}$. So $x^8$ is attained when $i = 4$, giving us a final answer of
		\begin{align*}
			{\frac12 \choose 4} \cdot 2^4 = \frac{\frac12 \cdot -\frac12 \cdot -\frac32 \cdot -\frac 52}{4!} \cdot 2^4 = -\frac{15}{4!}
		\end{align*}
	
		\item Notice that this sum equals $\sum_{k=0}^9 {9 \choose k} 9^k 1^{n-k} = (9+1)^9 = 10^9$.
		
		\item This is the hardest problem on the list. We want to show that
		\begin{align*}
			\sum_{k=1}^{n-1} k(n-k) = {n + 1 \choose 3}
		\end{align*}
		The answer is to partition on the middle number in the size 3 subset. For example, if the middle number is 2, 2 - 1 numbers can be below it, and $n+1 - (2)$ numbers can be above it. Similarly if the middle number is $j$ for $2 \leq j \leq n$, $j-1$ numbers can be below it and $n+1-j$ numbers can be above it, giving us the answer of
		\begin{align*}
			\sum_{j=2}^{n} (j-1)(n+1-(j)) = \sum_{j=1}^{n-1} j(n-j) = {n+1 \choose 3}
		\end{align*}
	
		\item We recall that $\sum_{k=1}^n {n \choose k} kx^k = xn(1+x)^{n-1}$. Plugging in $x = -1$ yields the required equation.
		
		\item We see that, if $a_n = p(n)$,
		\begin{center}
			\[\begin{tabular}{ >{$}l<{$} c c c c }
			 a_n & 2 & 3 & 6 & 12 \\
			 \Delta a_n & 1 & 3 & 6 \\
			 \Delta^2 a_n & 2 & 3 \\
			 \Delta^3 a_n & 1 \\
			\end{tabular}\]
		\end{center}
		This yields that $p(n) = 2 {n \choose 0} + 1 {n \choose 1} + 2 {n \choose 2} + 1 {n \choose 3} = 2 + n + n(n-1) + 1/6 n(n-1)(n-2)$. Plugging in $n = 1/2$ gives $2 + 1/2 + -1/4 + 1/6\cdot3/8 = 2 + 1/4 + 1/16 = 37/16$.
		
		\item This one is pretty cool. ${n \choose 2}^2$ is a fourth-degree polynomial, so we need the first 5 terms to do a difference table. $(n(n-1)/2)^2$ gives $0, 0, 1, 9, 36$. So the difference table is
		\begin{center}
			\[\begin{tabular}{ >{$}l<{$} c c c c c }
				a_n & 0 & 0 & 1 & 9 & 36 \\
				\Delta a_n & 0 & 1 & 8 & 27 \\
				\Delta^2 a_n & 1 & 7 & 19 \\
				\Delta^3 a_n & 6 & 12 \\
				\Delta^4 a_n & 6
			\end{tabular}\]
		\end{center}
		This gives a solution of $a_n = {n \choose 2} + 6 {n \choose 3} + 6{n \choose 4}$.
		
		\item Since $n \cdot 2^n$ is a solution, $2$ must be a double root of the characteristic equation, and similarly $3$ must be a root. This gives characteristic equation of $(q-3)(q-2)^2 = (q-3)(q^2-4q+4) = q^3 - 7q^2 + 16q - 12$. This gives us an answer of $a_n = 7a_{n-1} - 16a_{n-2} + 12$.
		
		\item Now for the fun ones. If $a_n$ is the number of views, we have that $a_1 = 1$, and $a_n = 3 + 2a_{n-1}$. This is a non-homogeneous linear equation, so a solution to the homogeneous version is just $b_n = A \cdot 2^n$, and we guess a particular solution to the nonhomogeneous version to be $c_n = C$, plugging this in gives $C = 3 + 2C$, i.e. $C = -3$. Adding these up gives us our final answer of $a_n = A \cdot 2^n - 3$. Plugging in our initial value gives us $A = 2$, so $a_n = 2^{n+1} - 3$.
		
		\item The final question. We used a very similar trick on the homework. We first pick the color of the furthest left tile, and then we pick if it is $1 \times 1$ or $1 \times 2$. In the first case we need to cover a $1 \times {n-1}$ board with the condition, and in the second we need to cover a $1 \times {n-2}$ Thus,
		\begin{align*}
			a_n = 2 \cdot (a_{n-1} + a_{n-2})
		\end{align*}
		This has characteristic equation $q^2 - 2q - 2 = 0$, which has roots $1 \pm \sqrt{2}$. So $a_n = C_1(1+\sqrt{2})^n + C_2(1-\sqrt{2})^n$. There is 1 way to tile a $1 \times 0$ rectangle trivially, and 2 ways to tile the $1 \times 1$ rectangle (one $1 \times 1$ tile and it is either red or blue). We get $a_0 = 1$ trivially, and $a_1 = 2$, so $a_2 = 6$, $a_3 = 16$, $a_4 = 44$, $a_5 = 120$, and $a_6 = 328$, and finally $a_7 = 896$.
	\end{enumerate}
\end{document}
